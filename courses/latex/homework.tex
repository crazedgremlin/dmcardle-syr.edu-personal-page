%  LaTeX document template for CIS 275 homework assignments.
%  Dan McArdle
%  10-14-2013

\documentclass[12pt]{article}
\usepackage[margin=1in]{geometry}
\usepackage{amsmath,amsthm,amssymb}

\begin{document}

% Boilerplate for creating a header
\title{Weekly Homework \#1}
\author{Dan McArdle\\
CIS 275 -- Discrete Math}
\date{Oct 14, 2013}
\maketitle  % without this line, the header would not be generated


% We're going to be writing a lot of backslashes, so let's avoid writing \backslash a lot by creating a macro.
\newcommand{\bs}{\backslash}

\section*{Problem 5}

\paragraph{Claim:} Let $A$, $B$, $C$ be sets. Then, $(A\bs B)\bs C \subseteq (A\bs C)\bs (B\bs C)$.

\paragraph{Proof: (direct)}

It is sufficient to show the following two subclaims.
\begin{enumerate}
	\item $(A\bs B)\bs C \subseteq (A\bs C)\bs (B\bs C)$
	\item $(A\bs B)\bs C \supseteq (A\bs C)\bs (B\bs C)$
\end{enumerate}

\paragraph{Subclaim 1:} $(A\bs B)\bs C \subseteq (A\bs C)\bs (B\bs C)$

Let $x$ be an arbitrary element in $(A\bs B)\bs C$.

It follows that $x\in A\bs B$ and $x \not \in C$.

Since $x \in A\bs B$, it follows that $x \in A$ and $x \not \in B$.\\


Since we know $x \in A$ and $x \not \in C$, it follows that $x \in A \bs C$.

Since we know that $x \not \in B$, it follows that $x \not \in B\bs C$.\\


Because $x \in A\bs C$ and $x \not \in B \bs C$, we have shown $x \in (A \bs C) \bs (B \bs C)$.

Therefore, since $x$ was an arbitrary element in $(A\bs B)\bs C)$, we have shown that $(A\bs B)\bs C \subseteq (A\bs C)\bs (B\bs C)$.  Subclaim 1 is true.



\paragraph{Subclaim 2:} $(A\bs B)\bs C \supseteq (A\bs C)\bs (B\bs C)$
[This is where we would prove that subclaim 2 is true. I leave it as an exercise for the reader!]

\paragraph{}
Because subclaims 1 and 2 are true, our original claim is true.

\end{document}